\documentclass{article}

\usepackage[utf8]{inputenc}
\usepackage{t1enc}
\usepackage[magyar]{babel}
\sloppy

\usepackage{amsmath}
\usepackage{comment}

\usepackage{minted}
\usepackage{graphicx}
\usepackage{caption}

\usepackage{amsthm}
\usepackage{amssymb}
\newtheorem{task}{feladat}
\newtheorem{lemm}{segédtétel}
\newtheorem{theorem}{tétel}
\newtheorem{answer}{válasz}
\newtheorem{conjecture}{sejtés}
\renewcommand{\qedsymbol}{$\blacksquare$}

\usepackage{cancel}

\newcommand{\squareparenthesed}[1]{\left[#1\right]}
\newcommand{\parenthesed}[1]{\left(#1\right)}
\newcommand{\openparenthesed}[1]{\left(#1\right.}
\newcommand{\closeparenthesed}[1]{\left.#1\right)}
\newcommand{\expvalnet}{E_{\text{netto}}}
\newcommand{\expvalgro}{E_{\text{brutto}}}

\newcommand{\setOf}[1]{\left\lbrace\,#1\,\right\rbrace}
\newcommand{\enumOf}[2]{\left\lbrace\,#1,\dots,#2\,\right\rbrace}
\newcommand{\setAbs}[2]{\left\lbrace\,#1\,\middle|\,#2\,\right\rbrace}

\usepackage{array}
\usepackage[table]{xcolor}
\newcommand{\blk}{\cellcolor{darkgray}}
\newcommand{\gry}{\cellcolor{lightgray}}
\newcommand{\red}{\cellcolor{red!33}}
\newcommand{\grn}{\cellcolor{green!33}}
\newcommand{\blu}{\cellcolor{blue!33}}
\newcommand{\ylw}{\cellcolor{yellow}}
\newcommand{\lyl}{\cellcolor{yellow!33}}
\newcommand{\myl}{\cellcolor{yellow!55}}

\author{Endrey Márk}
\title{Ész Ventura\\102.~feladvány\\Zsebtolvaj a ködben}

\begin{document}
	\maketitle

	\tableofcontents

	\section{Válaszok és a lényeg előrebocsátása}

	Azt érdemes mindjárt az elején észrevenni, hogy a feladat alapját képező sztochasztikus modell(ek) szempontjából a ,,rendőr'' és a ,,kapu'' szerepköre közt nincs különbség.
	A rendőr visszaadja a lefülelt pénzt az járókelőknek, a kapu nem, de a modellek szempontjából csak az  a lényeg, hogy mindkét objektum véget vet a játéknak.

	Ez alapján már válaszolhatunk is a feladat könnyebbik kérdésére:

	\begin{answer}
		Az, hogy a tolvaj elmenekül-e, vagy elfogatik, egyszerűen a kapu\-létszám és a rendőr\-létszám egymáshoz való arányától függ.
		Tehát pl.~ha az $a_K$ kapulétszám azonos az $a_R$ rendőrlétszámmal, akkor az esély 50\%-50\$.
		Precízebben ennek belátása a későbbi oldalakon megrajzolt \emph{eseményfákból} fog látszani.
	\end{answer}

	A sztochasztikus modellekben viszont a rendőrőknek és a kapuknak (ezeknek a ,,befejező objektumoknak'') már csak az $A := a_K + a_R$ összlétszáma fog szerepet játszani.
	A továbbiakban ezt az $A$ számot értelmezzük valamiféle általánosított értelemben vett rendőröknek. A kapu is ,,rendőr'' (csak nem adja vissza a pénzt az embereknek).

	Az $A$ paraméter mellett természetesen a másik fontos paraméter a sétáló emberek $N$ létszáma lesz.

	A fő kérdés mostantól tehát a ,,várható értéke'' a megszerzett pénznek, függetlenül attól, hogy azzal a tolvaj szabadon elmenekül, vagy lefülelik tőle.

	\begin{answer}
		Előrebocsátom, hogy ez a várható érték
		\[
			E_{A,N} = \frac N{A+1}
		\]
		ami a konrét feladat esetében $\frac{98}{4+1} = 19,6$.

		Azonban ez még nem a tulajdonképpeni feladatkérdésre adandó válasz, hiszen a feladat a tolvaj elmenekülésére kérdez rá (vagyis, amennyiben kapuk helyett rendőrőkre bukkan, a pénzt visszakapják a járókelők, így az nem számít ebbe bele).

		A pontos válaszhoz tekintsük úgy, hogy a kapuk is rendőrök --- különleges rendőrök ---, de a kapuk által ,,lefoglalt'' pénz egy különleges hivatalhoz kerül. Mennyi pénz kerül átlagosan ehhez a képzeletbeli ,,különleges hivatalhoz'', ha ismerjük az összes $A$ ,,rendőr'' által beszedett $E$ várható értéket? Nyilván a különleges rendőrök aránya a közönséges rendőrökhoz képest fog szerepet játszani ebben a fajta válaszpontosításban. Vagyis az előbb kapott $19,6$ értéknek a felét vesszük (hisz a kapuk és normál rendőrök száma megegyezik).

		Az ,,elmenekülős'' átlagos zsákmány tehát $9,8$. 
	\end{answer}

	\subsection{A numerikus út kudarca, vígaszként egy puszta sejtéssel}

	Mindenesetre a fentebb előrebocsátott

	\[
		E_{A,N} = \frac N{A+1}
	\]

	összefüggés elég meglepő egyszerűségű, az ember ennél nehézkesebb összefüggésekre gondolna elsőnek.

	Kis számpéldákra meg is nézhetjük rögtön.
	Azt rögtön láthatjuk, hogy az $A=0$ esetre a dolog triviálisan igaz: ha nincs sem kapu, sem rendőr, akkor $E_{A=0;N} = N$, a tolvaj egyszerűen kifoszt mindenkit, majd, miután mindenkit kifosztott lesz, szükségszerűen leáll, hisz nincs már kit kifosztani.

	Az $A=1$ esetén (egy ,,leállási objektum'', szóval kapu vagy rendőr jelenléte) a sejtés szerint a tolvaj az emberek felét fosztja ki átlagban: $E_{A=2;N} = \frac N2$. Képzeletben a tolvaj mintegy ,,megosztozik'' a leállási objektummal. Számszakilag nézve ugyanaz az eredmény, mintha kifosztana mindenkit, és megvesztegetve a rendőrt, átengedné neki a zsákmány felét!

	Ez az összefüggés eléggé meglepő egyszerűségi eredmény, főleg ha most előrebocsátom, milyen ,,nyers erőre'' alapozó számításos próbálkozások merültek fel elsőre a legtermészetesebb módon.
	Ezek ugyanis kissé elrejtik a lényeget, és eléggé bonyolultak!
	Ennek szemléltetésére egy leegyszerűsített feladatverzióhoz tartozó számítást kivonatosan előre is bocsátok:

	\begin{figure}[H]
		\caption*{Eseményfa az $E_{A=2;N=6}$ várható érték számítására: 6 járókelő, két ,,rendőr'':}
		\centering
		\includegraphics[scale=0.5]{draw-8=6+2}
	\end{figure}

	\begin{align*}
		E_{A:=2; N:=6} &= \frac{2}{8}0 + \frac{6}{8}\frac{2}{7}1 + \frac{6}{8}\frac{5}{7}\frac{2}{6}2 + \frac{6}{8}\frac{5}{7}\frac{4}{6}\frac{2}{5}3 + \frac{6}{8}\frac{5}{7}\frac{4}{6}\frac{3}{5}\frac{2}{4}4 + \frac{6}{8}\frac{5}{7}\frac{4}{6}\frac{3}{5}\frac{2}{4}\frac{2}{3}5 + \frac{6}{8}\frac{5}{7}\frac{4}{6}\frac{3}{5}\frac{2}{4}\frac{1}{3}\frac{2}{2}6\\
		  &= \frac{2}{8}0 + \frac{6}{8}\frac{2}{7}1 + \frac{\not6}{8}\frac{5}{7}\frac{2}{\not6}2 + \frac{\not6}{8}\frac{\not5}{7}\frac{4}{\not6}\frac{2}{\not5}3 + \frac{\not6}{8}\frac{\not5}{7}\frac{\not4}{\not6}\frac{3}{\not5}\frac{2}{\not4}4 + \frac{\not6}{8}\frac{\not5}{7}\frac{\not4}{\not6}\frac{\not3}{\not5}\frac{2}{\not4}\frac{2}{\not3}5 + \frac{\not6}{8}\frac{\not5}{7}\frac{\not4}{\not6}\frac{\not3}{\not5}\frac{\not2}{\not4}\frac{1}{\not3}\frac{2}{\not2}6\\
		  &= \frac2{8\cdot7}\parenthesed{6\cdot1 + 5\cdot2 + 4\cdot3 + 3\cdot2 + 2\cdot3 + 1\cdot6}\\
		  &= \frac2{8\cdot7}\cdot\frac{6^3+3\cdot6^2+2\cdot6}6 = 2
	\end{align*}

	Bár számszakilag kijött a megelőlegzett $E_{A,N} = \frac N{A+1}$ egyenlőség a konkrét értékekre, de rejtve maradt bármiféle általános összefüggés.

	Még nagyobb gond, hogy a számítás során szerepelt egy \[6\cdot1 + 5\cdot2 + 4\cdot3 + 3\cdot2 + 2\cdot3 + 1\cdot6\] összeg is, szóval egy ilyen ,,tolódószorzatos'' összeg is, amelyre nehéz skalázható módon zárt képletet találni.
	Bár egy érdekes tételcsalád (a ,,$\sum_{i=1}^n i^2$  ,,négyzetes'', a $\sum_{i=1}^n i^3$ ,,köbös'', \dots stb összegzési képletcsalád) segítségével felírhatóak zárt képletek az efféle tolódószorzatos összegekre is,
	de az eredeti feladat (4 ,,rendőr'', vagyis $E_{A=4;N=98}$ verzió) e segédtételek közül a \emph{negyedfokú} változatot fogja igényelni ($\sum_{i=1}^n i^4$), amely, bár több oldalnyi számolás után helyes eredményt fog adni, de teljesen rejtve fogja hagyni a végső egyszerű szép összefüggést. Arra jó, hogy maga a sejtés fel tud merülni, de csak a konkrét számolások numerikus eredményei által tud információt nyújtani: maga az összefüggés algebrailag nem jelenik meg benne. 

	Ez épp azért fájdalmas dolog, mert a végső összefüggés szinte kiabál egy reinterpretáció után.
	Nézzük csak meg jól a \[E_{A,N} = \frac N{A+1}\] sejtés képletét még egyszer!
	Olyan, mintha a tolvaj ,,osztozna'' a rendőrőkkel, megvesztegetve őket:
	\begin{itemize}
		\item Ha $A=1$ vagyis pontosan egyetlenegy rendőr van, akkor a tolvaj pont ,,felez'' vele.
		\item Ha pedig $A$ számú rendőr van, ,,igazságosan'' osztozik velük a tolvaj, és az elvileg maximális $N$ zsákmányból $\frac1{A+1}$ ,,osztalék'' szerint rá jutó részt, $\frac N{A+1}$ összeget irányoz elő magának.
	\end{itemize}


	Mindez már kínálja is a lényegi megoldási ötletet is: felépítek két különböző sztochasztikus modellt erre a feladatra.
	\begin{description}
		\item[Direkt modell:]
		Az egyik olyan lesz, mint  fentebb mutatott, a feladat leírásának természetesen megfelelő, de nehezen számolható modell.
		Előnye, hogy a feladat leírásának nyilvánvalóan pontosan megfelel, hátránya, hogy semmiben sem utal a végső $E_{A,N} = \frac N{A+1}$ összefüggésre, és nem is ad támpontot hozzá.
		A numerikus számolások lehetségesek, de nehézkesek lesznek. Bár sok végigszámolt sok példa után a sejtés felmerülhet, de a numerikus egyezésen túl semmi nem fog rá utalni.
		\item[Transzformált modell:]
		A másik sztochasztikus modell viszont olyan lesz, mintha direkt egyenesen a $E_{A,N} = \frac N{A+1}$ összefüggés felmutatására lenne megkonstuálva,
		és kifejezetten a rendőrök lefizetésére, illetve az e során megvalósuló ,,igazságos'' felosztásra épül.
		Viszont elsőre  egyáltalán nem fog látszani, hogy ennek a modellnek mi köze van egyáltalán a feladathoz
		(a végső összefüggést leszámítva, amit ugye még nem tudunk bizonyítani).
	\end{description}
	A bizonyítás lényege pontosan az lesz, hogy a két modell valamiféle rokonsága megmutatkozik. Ez a rokonság abban fog állni, hogy bizonyos értelemben mindkét modell egyfajta ,,igazságos játék'' megvalósulása lesz.

	\subsection{Direkt modell}

	Az első, a feladat leírása alapján felépített direkt modell egyszerűen az, hogy a rabló véletlenszerűen mozog.
	A többi szereplőt egyszerűség kedvéért állónak is képzelhetjük.
	Ha ,,befejezési objektumhoz'' ér hozzá (kapuhoz, rendőrhöz, egyszóval általánosított értelemben vett ,,rendőrhöz''), vége a játéknak.
	A felírható eseményfa, és a várható értéket kiszámoló képlet olyan jellegű, mint amilyet előbb az előrebocsátásnál felrajzoltam.

	Annyi --- a lényeget nem érintő --- interpretálási különbséget engedjünk meg, hogy
	\begin{itemize}
		\item Nincs kapu, csak rendőrök vannak. A kaput, ha szükség van á valamilyen részkérdéshez, tekintsük valamiféle megjelölt rendőrnek.
		\item Mindenki áll, csak a tolvaj mozog
		\item A rendőr az elkapáskor nem megfogja a tolvajt, hanem bevarrja az összes (még ki nem fosztott) járókelő zsebét, ezáltal a tolvaj számára elérhetetlenné teszi őket.
		De akár úgy is tekinthetjük, hogy a rendőr, ha találkozik a tolvajjal, kifosztja az összes intakt járókelőt, mintegy elébevágva a tolvajnak.
		A rendőr versengő tolvajtárs, de nem egyenrangú: viselkedése különleges a tolvajéhoz képest, mert a rendőr az összes ,,intakt'' járókelőt kifosztja, ha a tolvaj belébotlik.

		Eufémikusan interpretálhatjuk persze úgy, hogy ha a rendőrbe beleboltik egy tolvaj, a rendőr azonnal bemondja a hangosbeszélőben, hogy mindenki varrja be a zsebét, de a sztochasztikus modell szempontjából ez csak díszlet, a matematikai jelentése ugyanaz.
		\item A tolvaj alapvetően a fejében a járókelők vagyonának felét tartja normális zsákámynak (mármint 1 rendőr esetében; több rendőr esetén egyfajta $\frac1{A+1}$ ,,bribe osztalékhoz'' viszonyít). Mindenesetre a neki ,,járó'' vélt ,,jogos'' részesedésnél többet nyereségként, ennél kevesebbet pedig veszteségként él meg.
	\end{itemize}

	\subsection{Transzformált modell}

	A másik modell másfajta szcenáriót tartalmaz. Az egyszerűség kedvéért egyelőre legyen csak egy ,,befejezési objektum'', amit tekintsünk rendőrnek.
	A szabály az, hogy a tolvaj is áll, a rendőr is áll (és nem is lépnek soha közvetlen kapcsolatba egymással), viszont a járókelőket egyesével indítjuk útra, egész addig, amíg az épp aktuális járókelő véletlenszerűen választ a rendőr és a tolvaj között.

	A rendőr itt nem viselkedik különlegesen a tolvajhoz képest. Amennyiben egy járókelő őt választja a tolvajjal szemben, éppúgy kifosztja \emph{azt és csak azt az adott járókelőt}, mint ahogy a tolvaj tenné. A rendőr itt versengő és \emph{egyenrangú} tolvajtárs: a sztochasztikus modell szempontjából nyugodtan tekinthetjük úgy, hogy a rendőr éppúgy kifosztja a ,,neki jutó'' járókelőt, mint a tolvaj tenné, csak a maga számára.

	\subsection{Az ,,Adósok játéka'' interpretáció}
	
	A két modell nagyon más, máshogy néz ki az eseményfájuk, és azon egész más  valószínűségek és az értékek elrendezése is.
	De amíg mélyebb összefüggést nem találunk, addig is legalább a szöveges értelmezés szempontjából közelítsük a két modellt egymáshoz!

	Az $N$ járókelő mostantól nem más, mint adósok serege, akik összes vagyonukkal a tolvajnak és a rendőröknek tartoznak. A tolvaj valójában hitelező, sőt ő csak egy hitelező a többi, $A$ db hitelezőtársa mellett, akiket rendőröknek nevezünk. Minden hitelező (a tolvaj és a rendőrök) pontosan egyforma arányban adott kölcsön.

	Tehát pl. ha csak 1 a rendőrök összlétszáma, akkor a rendőr és a tolvaj teljesen egyenrangú hitelezőtársak, akik jogos igényt támasztanak a ,,járókelők'' (adósok) teljes vagyonára, fele-fele arányban minden egyes adós esetében.

	A két hitelező azonban egyfajta versengésben áll egymással: mindketten megpróbálnak jogos kölcsönrészükön felül is a lehető legtöbbet behajtani.
	Viszont legalább képesek megegyezni egymással a versengési játék szabályaiban. Ebben igazságosságra töekszenek: a játékban lehet nyerni és veszíteni, de a játékszabályoknak statisztikailag igazsgosaknak kell lenniük.

	\subsubsection{Az $A=1$ eset, egyetlen rendőr/hitelezőtárs}

	Most nézzük a két fent említett szotchasztikus modellt: vajon ezek rendre igazságos játéknak felelnek meg? A rendőrök száma egyelőre legyen pontosan 1.

	\paragraph{A transzormált modell} igazságossága nyilvánvaló:
	minden egyes adós esetében 
	\begin{itemize}
		\item 50\%, hogy a tolvaj ,,kapja meg'' az adóst, amely esetben övé az adós teljes vagyona, amely a jogos kölcsönrészt figyelembe véve $\frac12$ nyereséget jelent (hiszen ennyi többlethez jut az őt megillető jogos kölcsönrészen felül).
		\item Amennyiben a rendőr ,,kapja meg'' az adóst (szintén 50\% eséllyel), akkor az a tolvaj szempontjából $\frac12$ veszteséget jelent (hiszen elveszti az őt illető jogos kölcsönrészt).
	\end{itemize}
	A játék igazságos: azonosak az esélyek is, és a tétek is.

	\paragraph{A direkt modellben} a játék nem ilyen szimmetrikus, de ez is igazságos játéknak fog bizonyulni, mint alább meg fogjuk látni, csak ez nem ennyire nyilvánvaló.

	Továbbra is legyen csak egyetlen rendőr, és jelölje $n$ a hátralevő intakt járókelők számát,
	\begin{itemize}	
		\item $\frac{n}{n+1}$ az esély, hogy a tolvaj járókelőre (adósra) bukkan, amely esetben övé az adós teljes vagyona, amely a jogos kölcsönrészt figyelembe véve $\frac12$ nyereséget jelent (hiszen ennyi többlethez jut az őt megillető jogos kölcsönrészen felül).
		\item Amennyiben viszont a tolvaj rendőrre bukkan rá ($\frac1{n+1}$ eséllyel), a játék szabálya szerint a rendőr ki fogja fosztani az összes hátramaradt (vagyis $n$) járókelőt, ez pedig a a tolvaj szempontjából $\frac n2$ veszteséget jelent (az $n$ járókelő vagyonának a fele illetné meg őt jogos kölcsönrészként, márpedig a rendőr valamennyit kifosztja, azaz a tolvaj mind az $n$ játékos után egyszerre veszti el az őt illető jogos kölcsönrészt).
	\end{itemize}
	Azt sejtjük, hogy a nyilvánvaló asszimetria ellenére ez a játék is igazságos: a kis téthez nagy esély, a nagy téthez kis esély tartozik. A tolvaj csak kis eséllyel bukkan rendőrre, de akkor a rendőr javára nagy veszteség fogja érni őt.
	Számoljunk pontos várható értéket a nyereségre:
	\[E_{\text{lépés}} = \frac12\cdot\frac{n}{n+1} - \frac n2\cdot\frac1{n+1} = 0,\]
	vagyis a játék igazságos.

	Az $A = 1$ paraméter esetében tehát a két modell equivalens egymással a nyereség várható értéke szempontjából.

	\subsubsection{Az $A\in \mathbb N$ eset, tetszőleges számú rendőr/hitelezőtárs}

	Ugyanilyen megfontolással adódik, hogy a játék nemcsak $A = 1$, hanem tetszőleg $A$ számú ,,rendőr'' (hitelezőtárs) esetében is igazságos, természetesen ebben az esetben a tolvajnak nem $\frac12$-hez, hanem $\frac1{A+1}$ a jogos kölcsönrésze, ehhez kell viszonyítani a nyereséget és a veszteséget.

	\paragraph{A transzformált modell} igazságossága nyilvánvaló (a tétek és esélyek természetes megoszlása miatt). Az adósok véletlenszerűen ,,választanak'' a hitelezők között, akik közt a tolvaj is egyike a hitelezőknek. Mindegyik hitelező a ,,választóinak'' teljes vagyonát viszi. A direkt modelltől eltérően a rendőrök is ugyanolyan viselkedésű hitelezők mint a tolvaj: ,,megválasztásuk esetén'' ők is csak a saját választók vagyonát kapják.
	\begin{itemize}
		\item A tolvajnak minden egyes adós esetén $\frac1{A+1}$ az esélye, hogy őt válasszák, ekkor relatív nyeresége $1 - \frac{1}{A+1}$ lesz (hiszen a neki jogosan járó $\frac{1}{A+1}$ osztalék helyet az adós teljes vagyonát viszit, így épp ez a $1 - \frac{1}{A+1}$ különbözet a relatív haszna.
		\item A rossz eset a tolvaj szempontjából persze az (minden egyes adósra), ha az adós őhelyette valamelyik másik hitelezőt (vagyis egy rendőrt) választ. Ennek esélye $\frac A{A+1}$, --- va gyis elég nagy, --- viszont a veszteség mérsékelt: $\frac1{A+1}$ (vagyis a tolvaj elveszti az adós után neki járó jogos kölcsönrészt, de csak ennyit).
	\end{itemize}

	Látszik, hogy a kis tétekhez épp arányosan tartozik nagyobb esély, és viszont. Persze azért nézzük meg pontosan is: végezzük el a relatív nyereség várható értékének kiszámítását:
	\[E_{\text{lépés}} = \frac1{A+1}\cdot\underbrace{\parenthesed{1 - \frac{1}{A+1}}}_{\frac A{A+1}} - \frac A{A+1}\cdot\frac1{A+1} = 0.\]

	Vagyis a transzformált modell tetszőleges $A$ számú rendőr esetén is igazságos. Ez viszonylag jól látszik kapásból is, mert az esélyek és a tétek természetes módon felelnek meg egymásnak.

	\paragraph{A direkt modell} igazságosságának belátásához kevésbé nyilvánvaló most is, újra részletezzük a  várhatóérték-számítást:
	
	\begin{itemize}	
		\item $\frac{n}{n+A}$ az esély, hogy a tolvaj járókelőre (adósra) bukkan, amely esetben övé az adós teljes vagyona, amely a tolvajt illető jogos $\frac1{A+1}$ kölcsönrészhez viszonyítva relatívan  $1-\frac1{A+1}$ nyereséget jelent (hiszen ennyi többlethez jut az őt megillető jogos kölcsönrészen felül).
		\item Amennyiben viszont a tolvaj rendőrre bukkan rá ($\frac A{n+A}$ eséllyel), a játék szabálya szerint a rendőr ki fogja fosztani az összes hátramaradt (vagyis $n$) járókelőt, ez pedig a a tolvaj szempontjából $n\cdot\frac1{A+1} = \frac n{A+1}$ veszteséget jelent (az $n$ járókelő vagyonának a $\frac1{A+1}$ része illeti meg őt jogos kölcsönrészként, márpedig a rendőr valamennyit kifosztja, azaz a tolvaj mind az $n$ játékos után egyszerre veszti el az őt illető jogos kölcsönrészt).
	\end{itemize}
	Azt sejtjük, hogy a nyilvánvaló asszimetria ellenére ez a játék is igazságos: a kis téthez nagy esély, a nagy téthez kis esély tartozik. A tolvaj csak kis eséllyel bukkan rendőrre, de akkor a rendőr javára nagy veszteség fogja érni őt.
	Számoljunk pontos várható értéket a nyereségre egy járókelőnyi lépésre:
	\[E_{\text{lépés}} = \frac{n}{n+A}\cdot\underbrace{\parenthesed{1-\frac1{A+1}}}_{\frac A{A+1}} - \frac A{n+A}\cdot\frac n{A+1} = 0,\]
	vagyis a játék igazságos.	


	\section{Eseményfák. Modelltranszformáció nélküli ,,favágó'' számítások}

	Bár a fent említett ,,modellösszehasonlítós'' módszer nélkül, puszta számítási ,,nyers erővel'' is el lehet jutni a fő összefüggés
	\[
		E_{A,N} = \frac N{A+1}
	\]
	megsejtéséig, ez nem lesz bizonyító erejű. A nyers számítások mégsem értelmetlenek teljesen:
	például a hozzájuk tartozó ábrák --- az ,, \emph{eseményfák}'' --- sokat segítettek nekem a feladat megértésében,
	maguk a numerikus eredmények pedig a sejtés felállításában.

	Az eseményfákra próbálok általános sémát is mutatni, de főleg konkrét példák fognak szerepelni ,,kisebb'', ,,egyszerűsített'' feladatváltozatokra,
	amelyek abban különböznek az eredeti ,,nagy''feladattól, hogy kevesebb rendőr és/vagy  kevesebb járókelő van.

	\begin{figure}[H]
		\caption*{Eseményfa általános ábrája $E_{A,N}$ számításához}
		\centering
		\includegraphics[scale=0.5]{draw-general-rule}
	\end{figure}

	\begin{figure}[H]
		\caption*{Eseményfa $E_{A=1;N=6}$ számításához, jelezve a dolog algebráját is}
		\centering
		\includegraphics[scale=0.5]{draw-6+1}
	\end{figure}

	\begin{figure}[H]
		\caption*{Eseményfa $E_{A=1;N=6}$ számításához, numerikus értékekkel csak}
		\centering
		\includegraphics[scale=0.5]{draw-7=6+1}
	\end{figure}

	\begin{align*}
		E_{A:=1; N:=6} &= \frac{1}{7}0 + \frac{6}{7}\frac{1}{6}1 + \frac{6}{7}\frac{5}{6}\frac{1}{5}2 + \frac{6}{7}\frac{5}{6}\frac{4}{5}\frac{1}{4}3 + \frac{6}{7}\frac{5}{6}\frac{4}{5}\frac{3}{4}\frac{1}{3}4 + \frac{6}{7}\frac{5}{6}\frac{4}{5}\frac{3}{4}\frac{2}{3}\frac{1}{2}5 + \frac{6}{7}\frac{5}{6}\frac{4}{5}\frac{3}{4}\frac{2}{3}\frac{1}{2}\frac{1}{1}6\\
		  &= \frac{1}{7}0 + \frac{\not6}{7}\frac{1}{\not6}1 + \frac{\not6}{7}\frac{\not5}{\not6}\frac{1}{\not5}2 + \frac{\not6}{7}\frac{\not5}{\not6}\frac{\not4}{\not5}\frac{1}{\not4}3 + \frac{\not6}{7}\frac{\not5}{\not6}\frac{\not4}{\not5}\frac{\not3}{\not4}\frac{1}{\not3}4 + \frac{\not6}{7}\frac{\not5}{\not6}\frac{\not4}{\not5}\frac{\not3}{\not4}\frac{\not2}{\not3}\frac{1}{\not2}5 + \frac{\not6}{7}\frac{\not5}{\not6}\frac{\not4}{\not5}\frac{\not3}{\not4}\frac{\not2}{\not3}\frac{\not1}{\not2}\frac{1}{\not1}6\\
		  &= 0 + \frac17 + \frac27 + \frac37 + \frac47 + \frac57 + \frac67\\
		  &= \frac{1+2+3+4+5+6}7 = \frac{\frac{6\cdot7}2}7 = \frac{\frac{6\cdot\cancel7}2}{\cancel7} = 3
	\end{align*}

	\begin{figure}[H]
		\caption*{Eseményfa $E_{A=2;N=6}$ számításához, jelezve a dolog algebráját is}
		\centering
		\includegraphics[scale=0.5]{draw-6+2}
	\end{figure}

	\begin{figure}[H]
		\caption*{Eseményfa $E_{A=2;N=6}$ számításához, numerikus értékekkel csak}
		\centering
		\includegraphics[scale=0.5]{draw-8=6+2}
	\end{figure}

	\begin{align*}
		E_{A:=2; N:=6} &= \frac{2}{8}0 + \frac{6}{8}\frac{2}{7}1 + \frac{6}{8}\frac{5}{7}\frac{2}{6}2 + \frac{6}{8}\frac{5}{7}\frac{4}{6}\frac{2}{5}3 + \frac{6}{8}\frac{5}{7}\frac{4}{6}\frac{3}{5}\frac{2}{4}4 + \frac{6}{8}\frac{5}{7}\frac{4}{6}\frac{3}{5}\frac{2}{4}\frac{2}{3}5 + \frac{6}{8}\frac{5}{7}\frac{4}{6}\frac{3}{5}\frac{2}{4}\frac{1}{3}\frac{2}{2}6\\
		  &= \frac{2}{8}0 + \frac{6}{8}\frac{2}{7}1 + \frac{\not6}{8}\frac{5}{7}\frac{2}{\not6}2 + \frac{\not6}{8}\frac{\not5}{7}\frac{4}{\not6}\frac{2}{\not5}3 + \frac{\not6}{8}\frac{\not5}{7}\frac{\not4}{\not6}\frac{3}{\not5}\frac{2}{\not4}4 + \frac{\not6}{8}\frac{\not5}{7}\frac{\not4}{\not6}\frac{\not3}{\not5}\frac{2}{\not4}\frac{2}{\not3}5 + \frac{\not6}{8}\frac{\not5}{7}\frac{\not4}{\not6}\frac{\not3}{\not5}\frac{\not2}{\not4}\frac{1}{\not3}\frac{2}{\not2}6\\
		  &= \frac2{8\cdot7}\parenthesed{6\cdot1 + 5\cdot2 + 4\cdot3 + 3\cdot2 + 2\cdot3 + 1\cdot6}\\
		  &= \frac2{8\cdot7}\cdot\frac{6^3+3\cdot6^2+2\cdot6}6 = 2
	\end{align*}


	\begin{figure}[H]
		\caption*{Eseményfa $E_{A=4;N=10}$ számításához, jelezve a dolog algebráját is}
		\centering
		\includegraphics[scale=0.5]{draw-4+10}
	\end{figure}

	\begin{figure}[H]
		\caption*{Eseményfa $E_{A=4;N=10}$ számításához, numerikus értékekkel csak}
		\centering
		\includegraphics[scale=0.5]{draw-14=4+10}
	\end{figure}
	\begin{align*}
		&E_{A:=4; N:=10} = 0 + {}\\
		&+ \frac{10}{14}\frac{4}{13}1 + \frac{10}{14}\frac{9}{13}\frac{4}{12}2 + \frac{10}{14}\frac{9}{13}\frac{8}{12}\frac{4}{11}3 + \frac{10}{14}\frac{9}{13}\frac{8}{12}\frac{7}{11}\frac{4}{10}4 + \frac{10}{14}\frac{9}{13}\frac{8}{12}\frac{7}{11}\frac{6}{10}\frac{4}{9}5 + \frac{10}{14}\frac{9}{13}\frac{8}{12}\frac{7}{11}\frac{6}{10}\frac{5}{9}\frac{4}{8}6 {}+{}\\
		&+ \frac{10}{14}\frac{9}{13}\frac{8}{12}\frac{7}{11}\frac{6}{10}\frac{5}{9}\frac{4}{8}\frac{4}{7}7 + \frac{10}{14}\frac{9}{13}\frac{8}{12}\frac{7}{11}\frac{6}{10}\frac{5}{9}\frac{4}{8}\frac{3}{7}\frac{4}{6}8 + \frac{10}{14}\frac{9}{13}\frac{8}{12}\frac{7}{11}\frac{6}{10}\frac{5}{9}\frac{4}{8}\frac{3}{7}\frac{2}{6}\frac{4}{5}9 + \frac{10}{14}\frac{9}{13}\frac{8}{12}\frac{7}{11}\frac{6}{10}\frac{5}{9}\frac{4}{8}\frac{3}{7}\frac{2}{6}\frac{1}{5}\frac{4}{4}10 =\\
		&= \frac{10}{14}\frac{4}{13}1 + \frac{10}{14}\frac{9}{13}\frac{4}{12}2 + \frac{10}{14}\frac{9}{13}\frac{8}{12}\frac{4}{11}3 + \frac{\cancel{10}}{14}\frac{9}{13}\frac{8}{12}\frac{7}{11}\frac{4}{\cancel{10}}4 + \frac{\cancel{10}}{14}\frac{\cancel{9}}{13}\frac{8}{12}\frac{7}{11}\frac{6}{\cancel{10}}\frac{4}{\cancel{9}}5 + \frac{\cancel{10}}{14}\frac{\cancel{9}}{13}\frac{\cancel{8}}{12}\frac{7}{11}\frac{6}{\cancel{10}}\frac{5}{\cancel9}\frac{4}{\cancel8}6 {}+{}\\
		&+ \frac{\cancel{10}}{14}\frac{\cancel9}{13}\frac{\cancel8}{12}\frac{\cancel7}{11}\frac{6}{\cancel{10}}\frac{5}{\cancel9}\frac{4}{\cancel8}\frac{4}{\cancel7}7 + \frac{\cancel{10}}{14}\frac{\cancel9}{13}\frac{\cancel8}{12}\frac{\cancel7}{11}\frac{\cancel6}{\cancel{10}}\frac{5}{\cancel9}\frac{4}{\cancel8}\frac{3}{\cancel7}\frac{4}{\cancel6}8 + \frac{\cancel{10}}{14}\frac{\cancel9}{13}\frac{\cancel8}{12}\frac{\cancel7}{11}\frac{\cancel6}{\cancel{10}}\frac{\cancel5}{\cancel9}\frac{4}{\cancel8}\frac{3}{\cancel7}\frac{2}{\cancel6}\frac{4}{\cancel5}9 + \frac{\cancel{10}}{14}\frac{\cancel9}{13}\frac{\cancel8}{12}\frac{\cancel7}{11}\frac{\cancel6}{\cancel{10}}\frac{\cancel5}{\cancel9}\frac{\cancel4}{\cancel8}\frac{3}{\cancel7}\frac{2}{\cancel6}\frac{1}{\cancel5}\frac{4}{\cancel4}10 =\\
		&= \frac{10}{14}\frac{4}{13}1 + \frac{10}{14}\frac{9}{13}\frac{4}{12}2 + \frac{10}{14}\frac{9}{13}\frac{8}{12}\frac{4}{11}3 + \frac{9\cdot8\cdot7\cdot4}{14\cdot13\cdot12\cdot11}4 + \frac{8\cdot7\cdot6\cdot4}{14\cdot13\cdot12\cdot11}5 + \frac{7\cdot6\cdot5\cdot4}{14\cdot13\cdot12\cdot11}6 {}+{}\\
		&+ \frac{6\cdot5\cdot4\cdot4}{14\cdot13\cdot12\cdot11}7 + \frac{5\cdot4\cdot3\cdot4}{14\cdot13\cdot12\cdot11}8 + \frac{4\cdot3\cdot2\cdot4}{14\cdot13\cdot12\cdot11}9 + \frac{3\cdot2\cdot1\cdot4}{14\cdot13\cdot12\cdot11}10 =\\
		&= \frac{4}{14\cdot13\cdot12\cdot11}\openparenthesed{\boxed{12\cdot11\cdot10}\cdot1 + \boxed{11\cdot10\cdot9}\cdot2 + \boxed{10\cdot9\cdot8}\cdot3 + \boxed{9\cdot8\cdot7}\cdot4 + \boxed{8\cdot7\cdot6}\cdot5 +{}}\\
		&+ \closeparenthesed{\boxed{7\cdot6\cdot5}\cdot6 + \boxed{6\cdot5\cdot4}\cdot7 + \boxed{5\cdot4\cdot3}\cdot8 + \boxed{4\cdot3\cdot2}\cdot9 + \boxed{3\cdot2\cdot1}\cdot10} =\\
		&= \frac{4}{14\cdot13\cdot12\cdot11}\sum_{i=1}^{10} i\parenthesed{13-i}\parenthesed{12-i}\parenthesed{11-i} =\\
		&= \frac{4}{11\cdot12\cdot13\cdot14}\sum_{i=1}^{10} i\parenthesed{11-i}\parenthesed{12-i}\parenthesed{13-i} =\\
		\intertext{\dots próbáljuk meg általánosítani, paraméteresen felírni, ahol $N$ jelölje a sétálók, $A$ pedig a végkifejletek ($a_K$ kapuk és $a_R$ rendőrök együttes $a_K + a_R$) számát:}
		&= \frac A{\underbrace{\parenthesed{N+1} \cdot \cdots \cdot \parenthesed{N+A}}_{\text{Együtt $A$ darab tényező}}}\sum_{i=1}^N \overbrace{i\parenthesed{N+1-i}\parenthesed{N+2-i}\cdots\parenthesed{N+A-1-i}}^{\text{Együtt $A$ darab tényező, $1 +\parenthesed{A-1}$ megoszlásban}} =\\
		&= \frac A{\prod_{i=1}^A\parenthesed{N+i}}\sum_{i=1}^N i\prod_{j=1}^{A-1}\parenthesed{N+j-i}.\\
		\intertext{Azért az általánosítás mellett fejezzük be a konkrét példát is! Szóval ott tartottunk, hogy}
		\dots &= \frac{4}{11\cdot12\cdot13\cdot14}\sum_{i=1}^{10} i\parenthesed{11-i}\parenthesed{12-i}\parenthesed{13-i} ={}\\
		&= \frac{4}{11\cdot12\cdot13\cdot14}\cdot\frac{10^5 + 10\cdot10^4 + 35\cdot10^3 +50\cdot10^2 + 24\cdot10}{20} =\\
		&= 2.
	\end{align*}

	\begin{theorem}
		\[
			E_{A,N} = \frac A{\prod_{i=1}^A\parenthesed{N+i}}\sum_{i=1}^N i\prod_{j=1}^{A-1}\parenthesed{N+j-i}.
		\]
	\end{theorem}
	\begin{proof}
		A bizonyításnak elvileg a fent említett levezetésnél az általánosítási lépés átgondolásából kellene állnia,
		de itt most a ,,bizonyítás'' csak a korább említett kis számos példákra való ellenőrzésre fog korlátozódni.

		Amúgy még jó lenne programmal az eredeti szöveges példa alapján felépített, minnél fizikaközelibb, valósághűbb fizikai modellt felállítani
		(legalább egy stochasztikus modellt véletlenszámok erejéig),
		és nagyszámú futtatási kísérletre statisztikai ellenőrzést, hipotézisfelállítást végezni.

		Mindenesetre most a keretekhez igazodva szorítkozzunk csak a kisszámú példákra való sima numerikus ellenőrzésre!
		\begin{align*}
			E_{A=1; N=6 } &= \frac1{\parenthesed{6+1}}\sum_{i=1}^6 i = \frac1{\cancel{\parenthesed{6+1}}}\cdot\frac{6\cancel{\parenthesed{6+1}}}2 = 3\\
			E_{A=2; N=6 } &= \frac2{\parenthesed{6+1}\parenthesed{6+2}}\sum_{i=1}^6 i\parenthesed{6+1-i} = \frac2{\parenthesed{6+1}\parenthesed{6+2}}\cdot\frac{6^3+3\cdot6^2+2\cdot6}{6} = 2
			%E_{A=4; N=10} &= \frac4{11\cdot12\cdot13\cdot14}\sum_{i=1}^{10}i\parenthesed{11-i}\parenthesed{12-i}\parenthesed{13-i}
		\end{align*}
	\end{proof}

	\begin{theorem}
		\[
			E_{A=4;N} = \frac{N^5 + 10N^4 + 35N^3 +50N^2 + 24N}{5\parenthesed{N+1}\parenthesed{N+2}\parenthesed{N+3}\parenthesed{N+4}}.
		\]
	\end{theorem}
	\begin{proof}
		\begin{align*}
			E_{A=4; N} &= \frac4{\parenthesed{N+1}\parenthesed{N+2}\parenthesed{N+3}\parenthesed{N+4}}\sum_{i=1}^N i\parenthesed{N+1-i}\parenthesed{N+2-i}\parenthesed{N+3-i} =\\
			           &= \frac4{\parenthesed{N+1}\parenthesed{N+2}\parenthesed{N+3}\parenthesed{N+4}}\cdot\frac{N^5 + 10N^4 + 35N^3 +50N^2 + 24N}{20} =\\
			           &= \frac{N^5 + 10N^4 + 35N^3 +50N^2 + 24N}{5\parenthesed{N+1}\parenthesed{N+2}\parenthesed{N+3}\parenthesed{N+4}}
		\end{align*}
	\end{proof}


	\begin{figure}[H]
		\caption*{Eseményfa $4+98$ paraméterezéssel, jelezve a dolog algebráját is}
		\centering
		\includegraphics[scale=0.5]{draw-4+98}
	\end{figure}

	\begin{figure}[H]
		\caption*{Eseményfa $4+98$ paraméterezéssel, numerikus értékekkel csak}
		\centering
		\includegraphics[scale=0.5]{draw-102=4+98}
	\end{figure}

	\begin{answer}
		\begin{align*}
			E_{A=4;N=98} &= \frac{N^5 + 10N^4 + 35N^3 +50N^2 + 24N}{5\parenthesed{N+1}\parenthesed{N+2}\parenthesed{N+3}\parenthesed{N+4}} =\\
			             &= \frac{98^5 + 10\cdot98^4 + 35\cdot98^3 +50\cdot98^2 + 24\cdot98}{5\cdot99\cdot100\cdot101\cdot102}\\
			             &= \frac{98}5 = 19,6
		\end{align*}
	\end{answer}

	\begin{conjecture}
		\[
			E_{A,N} = \frac N{A+1}
		\]
	\end{conjecture}
	\begin{proof}
		Ez maga az a fő összefüggés, amit a korább említett kétféle sztochasztikus modell összevetésével próbáltam bizonyítani.
		És most az is be lett mutatva, hogy a mélyebb meggondolás nélküli ,,nyers erővel'' végzett numerikus számítások csak megsejteni engedték az összefüggést, de megragadni nem tudták.
	\end{proof}

	\section{Függelék}

	\subsection{Segédtétel}

	A fentiek során azonban felhasználtunk több segédtételt is.

	\subsubsection{Egy végtelen sorra vonatkozó segédtétel}
	\begin{lemm}
		\begin{equation}
			\sum_{i=0}^\infty ip^i = \frac p {\parenthesed{1-p}^2}
		\end{equation}
	\end{lemm}
	\begin{proof}
		\begin{align}
			\sum_{i=0}^n ip^i &= \sum_{i=1}^n ip^i\\
			\sum_{i=1}^n ip^i &= \sum_{i=1}^n p^i + \sum_{i=2}^n p^i + \sum_{i=3}^n p^i + \cdots + \sum_{i=n}^n p^i\\
			\sum_{i=m}^n p^i        &= \sum_{i=0}^{n-m} p^{m+i} = p^m \sum_{i=0}^{n-m}p^i\\
			\sum_{i=0}^{n-1}        &= \frac{p^n-1}{p-1}
		\end{align}
		\begin{align}
			\sum_{i=0}^\infty ip^i &= \sum_{i=1}^\infty ip^i\\
			\sum_{i=1}^\infty ip^i &= \sum_{i=1}^\infty p^i + \sum_{i=2}^\infty p^i + \sum_{i=3}^\infty p^i + \cdots\\
			\sum_{i=m}^\infty p^i  &= \sum_{i=0}^\infty p^{m+i} = p^m\sum_{i=0}^\infty p_i\\
			\sum_{i=0}^\infty      &= \frac1{1-p}
		\end{align}
		\begin{comment}
			\begin{align}
				\sum_{i=0}^\infty ip^i = \sum_{i=1}^\infty ip^i &= &  &    & &\sum_{i=1}^\infty p^i + {} &    &\sum_{i=2}^\infty p^i + {} &    &\sum_{i=3}^\infty p^i + \cdots\\
							                        &= &  &p^1 & &\sum_{i=0}^\infty p^i + {} & p^2&\sum_{i=0}^\infty p^i + {} & p^3&\sum_{i=0}^\infty p^i + \cdots\\
							                        &= & (&p^1 & &{}                    + {} & &p^2&{} + &p^3 + \cdots)\sum
			\end{align}
		\end{comment}
		\begin{align}
			\sum_{i=0}^\infty ip^i = \sum_{i=1}^\infty ip^i &= &    &\sum_{i=1}^\infty p^i + {} &    &\sum_{i=2}^\infty p^i + {} &    &\sum_{i=3}^\infty p^i + \cdots\\
					                                &= & p^1&\sum_{i=0}^\infty p^i + {} & p^2&\sum_{i=0}^\infty p^i + {} & p^3&\sum_{i=0}^\infty p^i + \cdots
		\end{align}
		\begin{multline}
			\sum_{i=0}^\infty ip^i = \parenthesed{p^1 + p^2 + \cdots}\sum_{i=0}^\infty p^i = \parenthesed{\sum_{i=1}^\infty p^i} \parenthesed{\sum_{i=0}^\infty p^i} = \parenthesed{p^1\sum_{i=0}^\infty p^i} \parenthesed{\sum_{i=0}^\infty p^i}=\\
			= p\parenthesed{\sum_{i=0}^\infty p_i}^2 = p\parenthesed{\frac1{1-p}}^2 = \frac p{\parenthesed{1-p}^2}
		\end{multline}
	\end{proof}

	\subsubsection{Tolódó-szorzatos szummázás segédtételcsaládja}

	\begin{lemm}
		\[\sum_{i=1}^n i\parenthesed{n+1-i} = \frac{n^3+3n^2+2n}{6}\]
	\end{lemm}
	\begin{proof}
		\begin{align}
			\sum_{i=1}^n i\parenthesed{n+1-i} = \sum_{i=1}^n\parenthesed{i\parenthesed{n+1} - i^2} &= \sum_{i=1}^n i\parenthesed{n+1} - \sum_{i=1}^n i^2\\
			                                                                                       &= \parenthesed{n+1}\sum_{i=1}^n i - \frac{2n^3+3n^2+n}{6}\\
			                                                                                       &= \parenthesed{n+1}\frac{n\parenthesed{n+1}}2 - \frac{2n^3+3n^2+n}{6}\\
			                                                                                       &= \frac{n\parenthesed{n+1}^2}2 - \frac{2n^3+3n^2+n}{6}\\
			                                                                                       &= \frac{n^3+3n^2+2n}{6}
		\end{align}
	\end{proof}

	\begin{lemm}
		\[
			\sum_{i=1}^N i\parenthesed{N+1-i}\parenthesed{N+2-i}\parenthesed{N+3-i} = \frac{N^5 + 10N^4 + 35N^3 +50N^2 + 24N}{20}
		\]
	\end{lemm}
	\begin{proof}
		\begin{align*}
			&\sum_{i=1}^N i\parenthesed{N+1-i}\parenthesed{N+2-i}\parenthesed{N+3-i} ={}\\
			&=  \sum_{i=1}^N\squareparenthesed{-i^4 + \parenthesed{3N+6}i^3 - \parenthesed{3N^2+12N+11}i^2 + \parenthesed{N^3+6N^2+11N+6}i}={}\\
			&= -\sum_{i=1}^N i^4 + \parenthesed{3N+6}\sum_{i=1}^N i^3 - \parenthesed{3N^2+12N+11}\sum_{i=1}^N i^2 + \parenthesed{N^3+6N^2+11N+6}\sum_{i=1}^Ni={}\\
			&= -\frac{6N^5+15N^4+10N^3-N}{30} + \parenthesed{3N+6}\squareparenthesed{\frac{N\parenthesed{N+1}}2}^2 - {}\\
			&\,\,\,\,\,\,\,- \parenthesed{3N^2+12N+11}\frac{2N^3+3N^2+N}6 + \parenthesed{N^3+6N^2+11N+6}\frac{N\parenthesed{N+1}}2 = {}\\
			&= \frac{N^5}{20} + \frac{N^4}2 + \frac{7N^3}4 + \frac{5N^2}2 + \frac{6N}5 = {}\\
			&= \frac{N^5 + 10N^4 + 35N^3 +50N^2 + 24N}{20}
		\end{align*}
	\end{proof}

	\subsubsection{Négyzetek szummája, köböké\dots\ stb jellegű segédtételcsalád}

	\begin{lemm}
		\[
			\sum_{i=1}^n i = \frac{\parenthesed{n+1}}2 = \frac{n^2+n}2
		\]
	\end{lemm}
	\begin{proof}
		Ld.~Gauß bizonyítását.
	\end{proof}

	\begin{lemm}
		\[
			\sum_{i=1}^n i^2 = \frac{2n^3+3n^2+n}6
		\]
	\end{lemm}
	\begin{proof}
		A polinom főegyütthatója (értéke $\frac13$) megsejthető a polinomok deriválási képletéből, pontosabban annak fordított jából, a primitív függvény képletéből.
		A többi együtthatóra ez nem ad támpontot, de ha a ,,plauzibilis'', esztétikailag valószínűsítehetően szóbajövő racionális számok egy véges halmazán számítógéppel próbákat végzünk
		\begin{align}
			\forall n &\in \enumOf07:\\
			\sum_{i=1}^n i^2 &= \frac{q_3n^3+q_2n^2+q_1n+q_0}6\\\intertext{ahol}
			q_3, q_2, q_1, q_0 &\in \setAbs{\frac ab\in\mathbb Q}{a \in \enumOf0{12}, b \in \enumOf1{12}}
		\end{align}
		Ez véges sok (bár nagyon sok) próbálkozás,
		a Haskell a fentiek alapján az alábbit adja
		\begin{align*}
			q_3 &:= \frac13\\
			q_2 &:= \frac12\\
			q_1 &:= \frac16\\
			q_0 &:= 0\\\intertext{ami épp a}
			\sum_{i=1}^n i^2 &= \frac{2n^3+3n^2+n}6\\\intertext{képlethez vezet.}
		\end{align*}
		Nem bizoníték, de ellenőrizzük kis mintán az egyezést:
		\begin{table}[H]
			\begin{tabular}{c|c|c||c}
				$n$	&	$n^2$	&	$\sum$	&	$\frac{2n^3+3n^2+n}6$\\\hline\hline
				0	&	0	&	0	&	0\\\hline
				1	&	1	&	1	&	1\\\hline
				2	&	4	&	5	&	5\\\hline
				3	&	9	&	14	&	14\\\hline
				4	&	16	&	30	&	30
			\end{tabular}
		\end{table}
	\end{proof}

	\begin{lemm}
		\[
			\sum_{i=1}^n i^3 = \frac{n^4+2n^3+n^2}4 = \squareparenthesed{\frac{n\parenthesed{n+1}}2}^2
		\]
	\end{lemm}
	\begin{proof}
		Mint az előbb, csak kicsit lassabban talál rá a gép, mert nagyobb a keresési tér.
		Ettől függetlenül másodpercek alatt elboldogul vele.
		A jövőre készülődve lehet a keresőprogramot idő szerint optimalizálni, illetve stratégialilag is javítani, és mindemellett akár gépközeli kódon is megírható a kereső algoritmus (pontosabban inkább heurisztika):
		\definecolor{bg}{rgb}{0.95,0.95,0.95}
		\begin{minted}[bgcolor=bg, fontsize=\footnotesize, tabsize=4]{c}
void sum3(void)
{
	printf("Sum of 3-powers\n");
	for (int i3 = 0; i3 < N; i3++) {
		for (int i2 = 0; i2 < N; i2++) {
			for (int i1 = 0; i1 < N; i1++) {
				bool flag = true;
				for (int n = 0; n < 7 && flag; n++) {
					int sum = 0;
					for (int i = 1; i <= n; i++) {
						sum += i * i * i;
					}
					float q4 = 1.0 / 4.0;
					float q3 = (float)sample[i3].a / (float)sample[i3].b;
					float q2 = (float)sample[i2].a / (float)sample[i2].b;
					float q1 = (float)sample[i1].a / (float)sample[i1].b;
					float q0 = 0.0;
					float val = q4*n*n*n*n + q3*n*n*n + q2*n*n + q1*n + q0;
					float d = val - (float)sum;
					d = d >= 0 ? d : -d;
					flag = flag && d < 0.01;
				}
				if (flag) {
					printf("q4 = %d/%d\n", 1, 4);
					printf("q3 = %d/%d\n", sample[i3].a, sample[i3].b);
					printf("q2 = %d/%d\n", sample[i2].a, sample[i2].b);
					printf("q1 = %d/%d\n", sample[i1].a, sample[i1].b);
					printf("q0 = %d\n", 0);
					printf("----------\n");
				}
			}
		}
	}
}
		\end{minted}
		A kódoláson túl két matematikai, tartalmi jellegű optimalizálást használtunk:
		\begin{itemize}
			\item A polinomba képzeletben 0-t elyettesítve tudjuk, hogy a konstans tag szükségszerűen 0.
			\item A főegyütthatóról nagyon erősen sejtjük, hogy $\frac15$, a primitív függvény és a határozatlan integrál számítási képletének analógiája alapján:
			\[\int x^n dx = \frac{x^{n+1}}{n+1} + \text{const}\]
		\end{itemize}
	\end{proof}

	\begin{lemm}
		\[
			\sum_{i=1}^n i^3 = \frac{n^4+2n^3+n^2}4 = \squareparenthesed{\frac{n\parenthesed{n+1}}2}^2
		\]
	\end{lemm}
	\begin{proof}
		Mint az előbb.
	\end{proof}


	\begin{lemm}
		\[
			\sum_{i=1}^n i^4 = \frac{6n^5+15n^4+10n^3-n}{30}
		\]
	\end{lemm}
	\begin{proof}
		Itt már \emph{nem} mint az előbb: nem használhatunk gépet, túlságosan nagy a keresési tér, lassú.
		A gép ettől függetlenül segített végül megtalálni a képletet, de nem favágó módszerrel.
		Afféle ,,félautomatizált'' megoldás vált be: sima Excel táblában ,,illeszetttm rá'' a szummázos adatsorra a polinomos adatsort próbálkozással, az együtthatókkal kísérletezve.
		Ha elég hosszú az Excel tábla példa-adatsora, ineraraktívan szépen megsejthető, mely irányban érdemes finomítani az együtthatókon, és éppen melyiket, hogy jobb illeszkedést kapjunk.
		Kell egy kis meggondolás hozzá, figyelni kell az eltérések előjelváltását és csökkenését, mintegyfajta ,,gradienses'' lemászást a hegyről.

		\setlength{\extrarowheight}{2px}
		\begin{table}[H]
			\caption*{Excel tábla: a $q_5, q_4, q_3, q_2, q_1, q_0 \in \mathbb Q$ együtthatók meghatározása\\a $\sum_{i=1}^ni^4 = q_5n^5+q_4n^4+q_3n^3+q_2n^2+q_1n^1+q_0n^0$ illesztése alapján}
			\begin{tabular}{|c|c|c|c|c|c|c|c|c|c|}
				\hline
				&&&&&&&&&\\\hline
				&$\gry q_5$&\ylw $\frac15$&&&&&&&\\\hline
				&$\gry q_4$&\ylw $\frac12$&&&&&&&\\\hline
				&$\gry q_3$&\ylw $\frac13$&&&&&&&\\\hline
				&$\gry q_2$&\ylw $0$&&&&&&&\\\hline
				&$\gry q_1$&\ylw $-\frac1{30}$&&&&&&&\\\hline
				&$\gry q_0$&\ylw $0$&&&&&&&\\\hline
				&&&&&&&&&\\\hline
				&&&&&&&&&\\\hline
				&&&\gry $n$ &	 \gry $n^4$      &$\gry \sum_{i=1}^ni^4$& \gry $q_5n^5+q_4n^4+q_3n^3+q_2n^2+q_1n^1+q_0n^0$&\blu\bf KÜLÖNBÖZET  &                 &             \\\hline
				&&&\lyl$0$ &\lyl     $0$      &\lyl       $0$       &\lyl       $0$&\myl$0$                                            &                 &             \\\hline
				&&&\lyl$1$ &\lyl     $1$      &\lyl       $1$          &\lyl       $1$&\myl$0$                                         &                 &             \\\hline
				&&&\lyl$2$ &\lyl    $16$      &\lyl      $17$           &\lyl      $17$&\myl$0$                                        &                 &             \\\hline
				&&&\lyl$3$ &\lyl    $81$      &\lyl      $98$          &\lyl      $98$&\myl$0$                                         &                 &             \\\hline
				&&&\lyl$4$ &\lyl   $256$      &\lyl     $354$          &\lyl     $354$&\myl$0$                                         &                 &             \\\hline
				&&&\lyl$5$ &\lyl   $625$      &\lyl     $979$          &\lyl     $979$&\myl$0$                                         &                 &             \\\hline
				&&&\lyl$6$ &\lyl  $1296$      &\lyl    $2275$         &\lyl    $2275$&\myl$0$                                          &                 &             \\\hline
				&&&\lyl$7$ &\lyl  $2401$      &\lyl    $4676$         &\lyl    $4676$&\myl$0$                                          &                 &             \\\hline
				&&&\lyl\vdots&\lyl\vdots      & \lyl\vdots         &    \lyl\vdots&\myl\vdots                                       &                 &             \\
			\end{tabular}
		\end{table}
		\setlength{\extrarowheight}{0px}
	\end{proof}

	\section{Emlékezet nélküli változat}

	Láttuk a feladat módosított, könnyített verzióit kisebb járokkelő- rendőr- és kapulétszámokra.
	Most nézzünk egy olyan módostott feladatot, amely az eddigiketől eltérő módon könnyít a feladaton!

	Most az legyen a feladat, hogy a tolvaj legyen ,,állapotmentes'', ne legyen ,,emlékezete'': legyen képes az általa már kirabolt embereket újra kirabolni.
	Ez persze egészen más  eredményhez fog vezetni, de érdemes látni a régi fogalmakat új kihívásban.

	\begin{figure}[H]
		\caption*{Eseményfa $p = \frac{98}{98+2+2}$ értékével}
		\centering
		\includegraphics{draw-static-choice-tree-3-levels}
	\end{figure}

	\subsection{Kérdések és válaszok}

	\begin{task}
		Mekkora a valószínűsége annak, hogy a tolvaj elmenekül?
	\end{task}

	\begin{answer}
		Válasz: 50\%.
	\end{answer}

	\begin{proof}
		A végtelen ciklusba ragadás valószínűsége az eseményfa alapján $\lim_{i\to\infty}p^i$, amely $0 < p < 1$ esetén $0$.
		A feladat szövegéből következően $p = \frac{98}{98+2+2}$, tehát megfelel a feltételnek, így a határárték tényleg 0.
		A végtelen ciklusba ragadás egyébként nem lehetetlen esemény, de valószínűsége 0 (és ez nem ellentmondás).

		A két másik lehetséges kimenetel valószínűsége egymással azonos --- ez szintén az eseményfából látszik.
		Tehát
		\begin{align}
			P\parenthesed{\text{Elmenekülés}} + P\parenthesed{\text{Elfogatás}} + P\parenthesed{\text{Végtelen}} &= 1\\
			P\parenthesed{\text{Végtelen}} &= \lim_{i\to\infty}p^i = 0\\
			P\parenthesed{\text{Elmenekülés}} : P\parenthesed{\text{Elfogatás}} &= 1 : 1\\
			P\parenthesed{\text{Elmenekülés}} : P\parenthesed{\text{Elfogatás}} : P\parenthesed{\text{Végtelen}} &= 1 : 1 : 0\\
			P\parenthesed{\text{Elmenekülés}} &= 0,5\\
			P\parenthesed{\text{Elfogatás}} &= 0,5
		\end{align}
	\end{proof}

	\begin{task}
		És ha elmenekül, akkor átlagosan hány ember kirablása után?
	\end{task}

	Még mielőtt a válaszon elgondolkodnánk, felejünk előbb egy olyan kérdésre, amelyet az eseményfa alapján könnyebb megválaszolni.

	Jelölje $\expvalgro$ a tolvaj által ,,átlagosan'' kirabolt emberek ,,bruttó'' számát (vagyis függetlenül attól, hogy végül a szabad utcára, vagy a rendőrök kezei közé kerül ezzel a pénzzel). 		Kontrasztként a ,,nettó'' jelentse azt, ha nem számoljuk be azokat az eseteket, amikor a rendőrök visszadják a pénzt a kárvallotaknak: az $\expvalnet$ netto érték legyen a sikeres elmenekülés során ,,átlagosan'', vagyis várható értékként meghatározható ,,netto'' rablásszám!

	Könnyű lenne rávágni, hogy $\expvalnet = \frac12\expvalgro$, és ez igaz is, de egyáltalán nem triviális. Ezért most először a $\expvalgro$ bruttó rablásszámot mondjuk meg! Ezt a  várható értéket az eseményfa és egy határérték-segédtétel alapján jól meghatározhatjuk:

	\begin{theorem}
		Az akár sikeres, akár rendőrökbe futó eseteket ,,átlagoló'', vagyis várható értékként meghatározható ,,bruttó'' rablásszám:
		\begin{equation}
			\expvalgro = \frac p{\parenthesed{1-p}^3},
		\end{equation}
	\end{theorem}
	\begin{proof}
		\begin{align}
			\expvalgro             &= \sum_{i=0}^\infty ip^i\parenthesed{1-p} = \parenthesed{1-p}\sum_{i=0}^\infty ip^i\\
			\sum_{i=0}^\infty ip^i &= \frac p{\parenthesed{1-p}^2}\\
			\expvalgro             &= \parenthesed{1-p} \cdot \frac p{\parenthesed{1-p}^2} = \frac p{\parenthesed{1-p}^3}
		\end{align}
	\end{proof}
	\begin{theorem}
		A bruttó és a nettó érték kapcsolata (e feladat esetében): $\expvalnet = \frac12\expvalgro$.
	\end{theorem}
	\begin{proof}
		A bizonyítás akár a feladat élő, ,,hétköznapi'' szövege alapján is elvégezhető, a ,,hétköznapi'' logika alapján, de természetesen formalizált bizonyítások is adhatóak.

		Az élőszavas bizonyítás: tegyük föl, hogy auz utcára kimenekült tolvajt is előbb-utóbb elkapja egy külön erre kiképzett különleges rendőrség.
		Végezzünk el sok független kísérlet, majd vessük össze a kétféle rendőrség által lefoglalt pénzeket!

		A formalizált bizonyítsás: végezzük el a végtelen sor képzését az eseményfából a megfelelő módosítással!
	\end{proof}

	\begin{answer}
		Az egyszerűsített feladat fenti kérdésére tehát a válasz: a ,,nettó'' trablásszám értéke, az pedig a fenti tételek felhasználásával
		\begin{equation}
			\expvalnet = \frac12\expvalgro = \frac12\cdot\frac p{\parenthesed{1-p}^3} = \frac p{2\parenthesed{1-p}^3}
		\end{equation}
	\end{answer}


	\subsection{Tesztek, különböző függetlenségű modellek}

	\subsubsection{Puszta numerikus ellenőrzési ,,modell''}

	\paragraph{A fő tesztfüggvény} Haskell-részlet:


	\definecolor{bg}{rgb}{0.95,0.95,0.95}
	\begin{minted}[bgcolor=bg]{haskell}
series:: Float -> Float
series p = sum $ take infinity $ zipWith (*) naturals (map (p **) naturals)

naturals = [0..]
infinity = 100
	\end{minted}

	\paragraph{A teljes tesztlési környezet} Tesztmintával együtt:

	\inputminted{haskell}{Series.hs}

	\subsubsection{,,Majdnem-fizikai'' modell}

	A ,,majdnem-fizikai'' modellnek az a feladata, hogy ne csak numberikusan ellenőrizze a felhasznált analízistételeket,
	hanem arra is adjn némi bizonyságot, hogy az alkalmazott tételek megválasztása, a bizonyítás alapgondolata jó-e.
	Ennek érdekében lényegében magát  feladat által leírt ,,fizikai világot'' kell minél közvetlenebbül fölépítenie,
	úgy, hogy a választ ,,közvetlenül'', a modellezett alapjelenségekből mintegy emergens viselkedéssel kapjuk  meg.

	Persze itt észszerű kompromisszumokat kötünk, és nem feltétlenül építünk véletlenszerűen bolyongó robotokat, sőt mégcsak a Brown-mozgást sem szimuláljuk.
	Megelégszünk annyi szintlemenéssel, hogy a véletlenszámok szintjén fogjuk modellezni a fealadatot, és megbízunk abban, hogy a Haskell által generált álvéletlenszámok legalább statisztikai értelemben eléggé ,,igaziak'',
\end{document}
